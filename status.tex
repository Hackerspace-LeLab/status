% 
% Author Steve Huguenin-Elie
% Legal Copyrights to Hackerspace Lelab
% Distribution Licensed CC BY-SA
% Description --
% Association status of Hackerspace Lelab
% Work based on Maj 2019 benevolat-vaud.ch
%
% Options for packages loaded elsewhere
\PassOptionsToPackage{unicode}{hyperref}
\PassOptionsToPackage{hyphens}{url}
%
\documentclass[10pt]{article}
\usepackage{lmodern}
\usepackage{amssymb,amsmath}
\usepackage{ifxetex,ifluatex}
\ifnum 0\ifxetex 1\fi\ifluatex 1\fi=0 % if pdftex
  \usepackage[T1]{fontenc}
  \usepackage[utf8]{inputenc}
  \usepackage{textcomp} % provide euro and other symbols
\else % if luatex or xetex
  \usepackage{unicode-math}
  \defaultfontfeatures{Scale=MatchLowercase}
  \defaultfontfeatures[\rmfamily]{Ligatures=TeX,Scale=1}
\fi
% Use upquote if available, for straight quotes in verbatim environments
\IfFileExists{upquote.sty}{\usepackage{upquote}}{}
\IfFileExists{microtype.sty}{% use microtype if available
  \usepackage[]{microtype}
  \UseMicrotypeSet[protrusion]{basicmath} % disable protrusion for tt fonts
}{}
\makeatletter
\@ifundefined{KOMAClassName}{% if non-KOMA class
  \IfFileExists{parskip.sty}{%
    \usepackage{parskip}
  }{% else
    \setlength{\parindent}{0pt}
    \setlength{\parskip}{6pt plus 2pt minus 1pt}}
}{% if KOMA class
  \KOMAoptions{parskip=half}}
\makeatother
\usepackage{xcolor}
\IfFileExists{xurl.sty}{\usepackage{xurl}}{} % add URL line breaks if available
\IfFileExists{bookmark.sty}{\usepackage{bookmark}}{\usepackage{hyperref}}
\hypersetup{
  hidelinks,
  pdfcreator={LaTeX via pandoc}}
\urlstyle{same} % disable monospaced font for URLs
\usepackage{graphicx}
\makeatletter
\def\maxwidth{\ifdim\Gin@nat@width>\linewidth\linewidth\else\Gin@nat@width\fi}
\def\maxheight{\ifdim\Gin@nat@height>\textheight\textheight\else\Gin@nat@height\fi}
\makeatother
% Scale images if necessary, so that they will not overflow the page
% margins by default, and it is still possible to overwrite the defaults
% using explicit options in \includegraphics[width, height, ...]{}
\setkeys{Gin}{width=\maxwidth,height=\maxheight,keepaspectratio}
% Set default figure placement to htbp
\makeatletter
\def\fps@figure{htbp}
\makeatother
\setlength{\emergencystretch}{3em} % prevent overfull lines
\providecommand{\tightlist}{%
  \setlength{\itemsep}{0pt}\setlength{\parskip}{0pt}}
\setcounter{secnumdepth}{-\maxdimen} % remove section numbering
\ifluatex
  \usepackage{selnolig}  % disable illegal ligatures
\fi
\usepackage[french]{babel}

\author{Steve Huguenin-Elie}
\date{\today}

\begin{document}

%\includegraphics[width=1.5in,height=1.15764in]{media/image1.jpeg}

\textbf{Statuts de Hackerspace Le Lab (\emph{avec commentaires})}

\textbf{Pour la création d'une Association (articles 60 et suivants CC)}

\begin{quote}
\textbf{Forme juridique, but et siège}

Art. 1

Sous le nom de Hackerspace Le Lab il est créé une Association à but non lucratif
régie par les présents statuts et par les articles 60 et suivants du
Code civil suisse.

\end{quote}
\begin{quote}
Art. 2

L'Association a pour but de proposer un espace collaboratif à profit des makers, startups, inventeurs, innovateurs dans le but de
\end{quote}
\begin{itemize}
  \tightlist
  \item
  \begin{quote}
    s'initier aux techniques contemporaines d'ingénierie (conception, production, assemblage) dans le respect des normes environnementales~;
  \end{quote}
  \item
  \begin{quote}
    trouver du soutien dans le développement d'une idée, concept en produit innovant à fort potentiel d'exploitation régionale~;
  \end{quote}
  \item
  \begin{quote}
    partager des connaissances et des techniques nouvelles liées à l'industrie microtechnique ou informatique.
  \end{quote}
\end{itemize}

\begin{quote}
Pour atteindre ce but, l'association développe notamment~:
\end{quote}
\begin{itemize}
  \tightlist
  \item \begin{quote}
    des partenariats avec les FABLABs, les Makerspaces, les Hackerspaces, les Chaos Computer Clubs constitués~;
  \end{quote}
  \item \begin{quote}
    des programmes de recherche autofinancés dont la publication, la participation et l'accès restent ouverts à tous~;
  \end{quote}
  \item \begin{quote}
    des solutions appliquées à l'environnement, à la santé et à la mobilité.
  \end{quote}
\end{itemize}

\begin{quote}
\emph{De par la loi, la transformation du but (ou mission) de
l'association requiert l'accord de chacun des membres de l'association.
Veillez donc à définir le(s) but(s) de l'Association avec précision mais
de manière non restrictive pour laisser de la souplesse à la poursuite
des objectifs et des moyens mis en œuvre pour les atteindre.}

\emph{A l'image de notre proposition, cet article est fréquemment
formulé en deux temps. Le premier décrit la mission dans les grandes
lignes et le second, plus pragmatique, évoque les principaux moyens mis
en œuvre par l'association pour atteindre son but.}

\end{quote}
\begin{quote}
Art. 3

Le siège de l'Association est au canton de Neuchâtel. Sa durée est illimitée.

\emph{Ne pas indiquer d'adresse postale afin de ne pas avoir à modifier
les statuts chaque fois qu'il y a un déménagement. Le nom d'une ville,
d'un canton ou la formule «~au domicile du président / trésorier
\ldots~» conviennent parfaitement.}

\emph{Il n'est pas obligatoire d'ouvrir le compte de votre association
là où celle-ci siège.}

\textbf{Organisation}

\end{quote}
\begin{quote}
Art. 4

Les organes de l'Association sont :
\end{quote}

\begin{itemize}
\item
  \begin{quote}
  l'Assemblée générale~;
  \end{quote}
\item
  \begin{quote}
  le Comité~;
  \end{quote}
\item
  \begin{quote}
  l'Organe de contrôle des comptes.
  \end{quote}
\end{itemize}

\begin{quote}
Art. 5

Les ressources de l'Association sont constituées par les cotisations
ordinaires ou extraordinaires de ses membres, des dons, ou legs, par des
produits des activités de l'Association et, le cas échéant, par des
subventions des pouvoirs publics.

L'exercice social commence le 1er janvier et se termine le 31 décembre
de chaque année.

Ses engagements sont garantis par ses biens, à l'exclusion de toute
responsabilité personnelle de ses membres.

\textbf{Membres}

\end{quote}
\begin{quote}
Art. 6

Peuvent être membres toutes les personnes ou organismes intéressés à la
réalisation des objectifs fixés par l'art. 2.

\emph{Si vous pensez créer/rédiger une charte qui récapitule les valeurs
et principes d'action de votre association, il peut être intéressant
d'en faire mention ici.}

\end{quote}
\begin{quote}
Art. 7

L'Association est composée de :
\end{quote}

\begin{itemize}
\item
  \begin{quote}
  membres individuels~;
  \end{quote}
\item
  \begin{quote}
  membres collectifs~;
  \end{quote}
\item
  \begin{quote}
  membres associés~.
  \end{quote}
\end{itemize}

\begin{quote}
\emph{Vous avez la possibilité d'envisager plusieurs catégories de
membres~comme des membres bienfaiteurs, membres de droit, membres
associés etc.}

\emph{Les membres collectifs peuvent être assimilés aux membres
individuels ou disposer d'un nombre de voix supérieur en fonction de
critères tels que leur taille, leur légitimité etc.}

\emph{Il est important de fixer dans les statuts les éventuelles
distinctions au niveau du droit de vote. (Cf. art. 14).}

\end{quote}
\begin{quote}
Art. 8

Les demandes d'admission sont adressées au Comité. Le Comité admet les
nouveaux membres et en informe l'Assemblée générale.

\end{quote}
\begin{quote}
Art. 9

La qualité de membre se perd :

a) par la démission. Dans tous les cas la cotisation de l'année reste
due.

b) par l'exclusion pour de ``~justes motifs~''.

L'exclusion est du ressort du Comité. La personne concernée peut
recourir contre cette décision devant l'Assemblée générale. Le non
paiement répété des cotisations (deux ans) entraîne l'exclusion de
l'Association.

\emph{La rédaction d'un article statutaire prévoyant l'exclusion d'un
membre pour de «~justes motifs~» a pour conséquence que le Juge n'a pas
le pouvoir de contrôler les motifs de la décision d'exclusion.}

\emph{Il convient de préciser dans les statuts la durée de non paiement
des cotisations susceptible d'entraîner l'exclusion.}

\end{quote}
\begin{quote}
\textbf{Assemblée générale}
Art. 10

L'Assemblée générale est le pouvoir suprême de l'Association. Elle
comprend tous les membres de celle-ci.

\end{quote}
\begin{quote}
Art. 11

Les compétences de l'Assemblée générale sont les suivantes. Elle :
\end{quote}

\begin{itemize}
\item
  \begin{quote}
  adopte et modifie les statuts~;
  \end{quote}
\item
  \begin{quote}
  élit les membres du Comité et de l'Organe de contrôle des comptes~;
  \end{quote}
\item
  \begin{quote}
  détermine les orientations de travail et dirige l'activité de
  l'Association~;
  \end{quote}
\item
  \begin{quote}
  approuve les rapports, adopte les comptes et vote le budget~;
  \end{quote}
\item
  \begin{quote}
  donne décharge de leur mandat au Comité et à l'Organe de contrôle des
  comptes~;
  \end{quote}
\item
  \begin{quote}
  fixe la cotisation annuelle des membres individuels et des membres
  collectifs~;
  \end{quote}
\item
  \begin{quote}
  prend position sur les autres projets portés à l'ordre du jour~:
  \end{quote}
\end{itemize}

\begin{quote}
L'Assemblée générale peut saisir ou être saisie de tout objet qu'elle
n'a pas confié à un autre organe.

\end{quote}
\begin{quote}
Art. 12

Les assemblées sont convoquées au moins 20 jours à l'avance par le
Comité par le biais du courrier électronique. Le Comité peut convoquer 
des assemblées générales extraordinaires aussi souvent que le besoin 
s'en fait sentir.

\emph{Sans précision, on considère que l'invitation est transmise par
courrier écrit. Si vous souhaitez, dans la mesure du possible, convoquer
les AG par courrier électronique, nous vous recommandons d'en faire
mention ici.}

\end{quote}
\begin{quote}
Art. 13

L'assemblée est présidée par le Président(e) ou un autre membre du Comité.

Le secrétaire de l'Association ou un autre membre du comité tient le
procès-verbal de l'Assemblée~; il le signe avec le président. \emph{il
le signe avec le président. u comité tient le proclurs, n'unication
d'tion de l'hacune d'ès-verbal}

\end{quote}
\begin{quote}
Art. 14

Les décisions de l'Assemblée générale sont prises à la majorité simple
des membres présents. En cas d'égalité des voix, celle du (de la) président(e) est
prépondérante.

Les décisions relatives à la modification des statuts ne peuvent être
prises qu'à la majorité des 2/3 des membres présents.

\emph{Sans précision dans les statuts, les décisions sont prises à la
majorité des voix des membres présents. Les abstentions et les suffrages
nuls sont considérés comme des votes négatifs, car ils sont pris en
compte pour le calcul de la majorité~; il est donc possible de prévoir
ici que les abstentions ne comptent pas comme des votes négatifs.}

\emph{Il est possible de prévoir que les décisions sont prises à la
majorité absolue (+ de 50\%) ou à la majorité relative (ce qui obtient
le plus de voix). A défaut d'indication, l'application de la majorité
relative est préconisée (plus simple et plus rapide); dans un tel cas,
il devra en être fait mention lors de la convocation de l'AG.~}

\emph{Pour certains objets, les statuts peuvent aussi prévoir une
majorité qualifiée (2/3 ou 3/4) ou un quorum de présence.}

\emph{Si votre Association est une faîtière, indiquez le nombre de voix
qui va être accordé aux sections. Par exemple, vous pouvez écrire `` Les
sections locales disposent de 10 voix jusqu'à 50 membres~; 20 voix
jusqu'à 100 membres ''.}

\end{quote}
\begin{quote}
Art. 15

Les votations ont lieu à main levée. À la demande de 5 membres au moins,
elles auront lieu au scrutin secret. Le vote par production est admis.

\emph{Les statuts peuvent admettre le vote par procuration. Par
exemple~ainsi «~Les membres absents ont la possibilité de donner
procuration à l'un des membres de l'Association. Toutefois le
représentant ne peut recevoir plus de deux procurations.~»}

\end{quote}
\begin{quote}
Art. 16

L'Assemblée se réunit au moins une fois par an sur convocation du
Comité.

\end{quote}
\begin{quote}
Art. 17

L'ordre du jour de cette assemblée annuelle (dite ordinaire) comprend
nécessairement :
\end{quote}

\begin{itemize}
\item
  \begin{quote}
  le rapport du Comité sur l'activité de l'Association pendant l'année
  écoulée~;
  \end{quote}
\item
  \begin{quote}
  un échange de points de vue/décisions concernant le développement de
  l'Association~;
  \end{quote}
\item
  \begin{quote}
  les rapports de trésorerie et de l'Organe de contrôle des comptes~;
  \end{quote}
\item
  \begin{quote}
  l'élection des membres du Comité et de l'Organe de contrôle des
  comptes~;
  \end{quote}
\item
  \begin{quote}
  les propositions individuelles.
  \end{quote}
\end{itemize}

\begin{quote}
Art. 18

Le Comité est tenu de porter à l'ordre du jour de l'Assemblée générale
(ordinaire ou extraordinaire) toute proposition d'un membre présentée
par écrit au moins 10 jours à l'avance.

\emph{Si vous décidez de modifier le délai fixé ici à 10 jours, veillez
à modifier l'article 12 en conséquence.}

\end{quote}
\begin{quote}
Art. 19

L'Assemblée générale extraordinaire se réunit sur convocation du Comité
ou à la demande d'un cinquième des membres de l'Association.

\emph{Ici, la valeur d'un 1/5 des membres est imposée par la loi. Il est
par exemple possible de prévoir un nb moins important de membres, même
un seul membre p.ex.}

\textbf{Comité}

\end{quote}
\begin{quote}
Art. 20

Le Comité exécute et applique les décisions de l'Assemblée générale. Il
conduit l'Association et prend toutes les mesures utiles pour que le but
fixé soit atteint. Le Comité statue sur tous les points qui ne sont pas
expressément réservés à l'Assemblée générale.

\end{quote}
\begin{quote}
Art. 21

Le Comité se compose au minimum de trois membres, nommés pour deux ans
par l'Assemblée générale.

\emph{Le «~minimum~» dépend des caractéristiques et notamment de la
taille de votre association. Privilégiez un nombre impair.}

\emph{La durée idéale des mandats dépend de nombreux facteurs tels que
la complexité des activités et de l'organisation de l'association, du
nombre de membres soit des potentiels remplaçants au comité etc.}

\emph{Limiter le nombre de réélection oblige à terme le renouvèlement du
comité (vous n'êtes pas tenus de délimiter le nombre de réélections
possibles).}

\end{quote}
\begin{quote}
Art. 22

Le Comité se constitue lui-même. Il se réunit autant de fois que les
affaires de l'Association l'exigent. Le comité délibère valablement,
quel que soit le nombre des membres présent-e-s. Il prend ses décisions
à la majorité simple des membres présent-e-s

\emph{«~se constitue lui-même~» signifie que les membres de cet organe
vont décider eux-mêmes de la façon dont ils vont se répartir les
différentes fonctions (présidence, secrétariat, comptabilité, etc.).
Cela peut faciliter la prise de responsabilités lorsqu'il n'est pas
possible d'élire un président lors de l'Assemblée générale, faute de
candidats disposés à assumer cette fonction.}

\emph{Si vous souhaitez néanmoins que ce soit l'AG qui élise
individuellement le président et/ou les autres membres / fonctions du
comité, veillez à modifier les articles 11 et 17.}

\end{quote}
\begin{quote}
Art. 23~~~

En cas de vacance en cours de mandat, le Comité peut se compléter par
cooptation jusqu'à la prochaine assemblée générale.

Si la fonction de Président(e) devient vacante, le (la)
vice-Président(e) ou un autre membre du Comité lui succède jusqu'à la
prochaine assemblée générale.

\end{quote}
\begin{quote}
Art. 24

L'Association est valablement engagée par la signature collective de
deux membres du Comité.

\end{quote}
\begin{quote}
Art. 25

Le Comité est chargé :
\end{quote}

\begin{itemize}
\item
  \begin{quote}
  de prendre les mesures utiles pour atteindre les objectifs visés~;
  \end{quote}
\item
  \begin{quote}
  de convoquer les assemblées générales ordinaires et extraordinaires~;
  \end{quote}
\item
  \begin{quote}
  de prendre les décisions relatives à l'admission et à la démission des
  membres ainsi qu'à leur exclusion éventuelle~;
  \end{quote}
\item
  \begin{quote}
  de veiller à l'application des statuts, de rédiger les règlements et
  d'administrer les biens de l'Association.
  \end{quote}
\end{itemize}

\begin{quote}
Art. 26

Le Comité est responsable de la tenue des comptes de l'Association.

\end{quote}
\begin{quote}
Art. 27

Le Comité engage (licencie) les collaborateurs salariés et bénévoles de
l'Association. Il peut confier à toute personne de l'Association ou
extérieure à celle-ci un mandat limité dans le temps.

\emph{Si des salariés sont engagés, ils peuvent être invités à
participer aux travaux du Comité avec une voix consultative. Dans les
grandes Associations, le personnel salarié peut se faire représenter par
une ou deux personnes disposant d'une voix délibérative. Le cas échéant,
veillez à préciser ces modalités dans les statuts.}

\end{quote}
\begin{quote}
\textbf{Organe de contrôle}

Art. 28

L'organe de contrôle des comptes vérifie la gestion financière de
l'Association et présente un rapport à l'Assemblée générale. Il se
compose de deux vérificateurs élus par l'Assemblée générale.

\end{quote}
\begin{quote}
\textbf{Dissolution}

Art. 29

La dissolution de l'Association est décidée par l'Assemblée générale à
la majorité des deux tiers des membres présents. L'actif éventuel sera
attribué à un organisme se proposant d'atteindre des buts analogues.

Les présents statuts ont été adoptés par l'assemblée constitutive
du\ldots{} à\ldots{}

Au nom de l'Association

Président: M. Huguenin-Elie Steve

Vice-présidente: Mme Aronoff Rachel, PhD
\end{quote}

%\includegraphics[width=1.95833in,height=0.94792in]{media/image2.jpeg}

\end{document}
